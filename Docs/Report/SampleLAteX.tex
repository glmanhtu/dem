\documentclass[11pt]{article}
%Gummi|065|=)
\title{\textbf{Generation of a 3D object from a Digital Elevation Model (DEM)}}
\author{VU Manh Tu\\
		NGUYEN Bao Xuan Truong\\
		HO Do Quynh Huong\\
		PHAM Phu Quoc\\
		BUI The Anh
		}
\date{07-04-2016}
\begin{document}

\maketitle

\section{Abstract}

Digital Elevation Models (DEMs) are 2D maps in which each point is associated with its height.They can be obtained through techniques such as \emph{photogrammetry} \footnote{https://somesite.net} ,\emph{lidar} \footnote{http://somesite.net},\emph{land} \footnote{http://somesite.net},\emph{surveying} \footnote{http://somesite.net},\emph{etc} \footnote{http://somesite.net}.They represent the elevation of a terrain map. 

\section{Bibliography}

For this project, the first thing to do is to do a state-of-the-art of DEMs, that is, to try to \emph{classify DEMs models (stereography, satellite, etc.)} and to \emph{identify the differents formats}\footnote{http://www.ngdc.noaa.gov/mgg/dem/}.
The Project will be implement in \emph{C++/ Qt framework}\footnote{http://http://www.qt.io/ide/} and a visualisation in \emph{OpenGL}\footnote{opengl-superbible-comprehensive-tutorial-and-reference-5th-edition-2010}.

\end{document}
